\documentclass[a4paper,10pt]{article}

\usepackage[utf8]{inputenc}
\usepackage[french]{babel}
\usepackage[french,ruled,vlined,linesnumbered]{algorithm2e}
\usepackage{amsmath,amssymb,amsfonts,amsthm}
\usepackage{hyperref}
\hypersetup{colorlinks=true,urlcolor= blue,linkcolor= black}

\newcommand*{\itemb}{\item[$\bullet$]}
\newcommand*{\itemt}{\item[$\blacktriangleright$]}
\DeclareMathOperator{\degree}{degre}
\DeclareMathOperator{\taille}{taille}
\DeclareMathOperator{\voisin}{voisin}
\DeclareMathOperator{\voisins}{voisins}
\DeclareMathOperator{\degmax}{degmax}
\DeclareMathOperator{\argmax}{argmax}
\DeclareMathOperator{\argmin}{argmin}
\DeclareMathOperator{\cover}{cover}
\DeclareMathOperator{\nonempty}{nonempty}
\DeclareMathOperator{\true}{true}
\DeclareMathOperator{\false}{false}
\DeclareMathOperator{\leaf}{leaf}
\DeclareMathOperator{\parent}{parent}
\DeclareMathOperator{\Color}{color}
\DeclareMathOperator{\degrepositif}{degrePositif}

\newcommand*{\dlor}[2]{\displaystyle{\underset{#1}{\overset{#2}{\bigvee}}}}
\newcommand*{\dland}[2]{\displaystyle{\underset{#1}{\overset{#2}{\bigwedge}}}}

\title{Projet Complexité et Algorithmique appliquée\\ Implémentation d'algorithmes pour résoudre le problème Vertex Cover \\
README}
\author{Thomas \bsc{Bellitto}\and Paul-Émile \bsc{Boutoille} \and Claire \bsc{Pennarun}}
\date{}

\begin{document}

\maketitle
\section{Fichiers présents}

	Le répertoire "src" contient les fichiers source .c
	
	Le répertoire "include" contient les fichiers header .h
	
	Le répertoire "files" contient les fichiers .txt représentant des graphes.
	
	Attention, les fichiers de graphes doivent être au format .txt et au format suivant :

	nbSommets
	
	indiceSommet1 : indiceVoisin1 indiceVoisin2 ...
	
	indiceSommet2 : indiceVoisin1 indiceVoisin2 ...
	
	Le répertoire "tests" contient les fichiers .c de tests.
	
	A la racine sont présents les exécutables et le Makefile.


\section{Installation et compilation}

	Pour utiliser la résolution avec \texttt{minisat}, il faut installer ce logiciel.
	Pour l'installer sous Linux :
	\texttt{sudo apt-get install minisat}
	
	Nous ne sommes pas sûrs que les appels système utilisés pour résoudre le problème VertexCover à l'aide de \texttt{minisat} soient reconnus sous d'autres systèmes d'exploitation que GNU/Linux.

	Les fichiers se compilent avec la commande \texttt{make} qui lance l'exécution du fichier Makefile présent à la racine et génère l'exécutable \textit{vertexCoverExec}.


\section{Exécution}

	L'exécutable \textit{vertexCoverExec} se lance avec la commande \texttt{./vertexCoverExec}, en sélectionnant l'algorithme que l'on veut lancer et le graphe sur lequel on veut le faire :
	
	\bigskip
	\texttt{./vertexCoverExec [algorithme] [graphe] ([disp])}

	\bigskip
	\texttt{[algorithme]} : algorithme à lancer, avec ses paramètres
	\begin{itemize}
		\item Algorithme glouton : \texttt{greedy}
		\item Algorithme optimal pour les arbres : \texttt{treeOpt}
		\item Algorithme optimal pour les graphes bipartis : \texttt{bipartiteOpt}
		\item Algorithme 2-approché avec arbre couvrant : \texttt{spanningTree}
		\item Algorithme 2-approché avec élimination d'arêtes : \texttt{edgesDeletion}
		\item Algorithme optimal pour petite couverture : \texttt{littleCover} + $tailleCouverture$
		\item Pour tester l'existence d'une couverture de taille $k$ sur un graphe avec \texttt{minisat} : \texttt{minisat k}
	\end{itemize}
	
	\bigskip
	\texttt{[graphe]} : graphe sur lequel exécuter l'algorithme
	\begin{itemize}
		\item \texttt{chemin/vers/le/graphe.txt}
		\item OU appel à une fonction de génération aléatoire avec la taille du graphe et la probabilité d'apparition d'une arête :
			\begin{itemize}
			\itemb Génération d'arbres de taille $n$ : \texttt{treeGen} + $n$
			\itemb Génération de graphes de taille $n$ et proba $p$ : \texttt{gen} + $n$ + $p$
			\itemb Génération de graphes bipartis de taille $n$ et proba $p$ : \texttt{bipartiteGen} + $n$ + $p$
			\itemb Génération de graphes de taille $n$, proba $p$ et taille de couverture $k$ : \texttt{littleGen} + $n$ + $k$ + $p$
			
			\end{itemize}
	\end{itemize}
	
	\bigskip
	\texttt{[disp]} : argument facultatif, permettant d'afficher le graphe sur lequel l'algorithme est exécuté (pas disponible avec minisat).
	
	\paragraph{Exemples}
	\begin{itemize}
	\item Pour lancer l'algorithme glouton sur un graphe de 17 sommets généré aléatoirement avec une couverture de 5 sommets et une probabilité de 0.7 :
	\texttt{./vertexCoverExec greedy littleGen 17 5 0.7}
	
	\item Pour lancer chercher une couverture de taille 4 sur un arbre de taille 12 généré aléatoirement :
	\texttt{./vertexCoverExec littleCover 4 treeGen 12}
	\end{itemize}
	
	\paragraph{Options}
	\begin{itemize}
	\item Pour connaître les commandes possibles : \texttt{./vertexCoverExec help}
	
	\end{itemize}

\section{Tests}
	Les tests peuvent être lancés avec l'exécutable \textit{testExec}, qui est généré grâce à la commande \texttt{make testTime}. Il faut spécifier le nombre de graphes sur lesquels on veut lancer les tests, leur fonction de génération et l'algorithme à tester :
	
	\bigskip
	\texttt{./testExec [nbGraphes] [generation] [algorithme]}
	
	\bigskip
	\texttt{[nbGraphes]} : nombre de graphes sur lesquels on fait les tests
	
	\bigskip
	\texttt{[generation]} : choix de l'algorithme de génération aléatoire et paramètres 
	\begin{itemize}
			\itemb Génération d'arbres de taille $n$ : \texttt{treeGen} + $n$
			\itemb Génération de graphes de taille $n$ et proba $p$ : \texttt{gen} + $n$ + $p$
			\itemb Génération de graphes bipartis de taille $n$ et proba $p$ : \texttt{bipartiteGen} + $n$ + $p$
			\itemb Génération de graphes de taille $n$, proba $p$ et taille de couverture $k$ : \texttt{littleGen} + $n$ + $k$ + $p$
	\end{itemize}
	
	\bigskip
	\texttt{[algorithme]}
	\begin{itemize}
		\item Algorithme glouton : \texttt{greedy}
		\item Algorithme optimal pour les arbres : \texttt{treeOpt}
		\item Algorithme optimal pour les graphes bipartis : \texttt{bipartiteOpt}
		\item Algorithme 2-approché avec arbre couvrant : \texttt{spanningTree}
		\item Algorithme 2-approché avec élimination d'arêtes : \texttt{edgesDeletion}
		\item Algorithme optimal pour petite couverture : \texttt{littleCover} + $tailleCouverture$
	\end{itemize}
	

\end{document}
