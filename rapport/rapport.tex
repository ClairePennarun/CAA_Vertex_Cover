\documentclass[a4paper,10pt]{article}
\usepackage[utf8]{inputenc}
\usepackage[french]{babel}
\usepackage[utf8]{inputenc}
\title{Projet CAA : couverture par sommets}
\author{Thomas \bsc{Bellitto}\and Paul-Émile \bsc{Boutoille} \and Claire \bsc{Pennarun}}
\date{}

\begin{document}

\maketitle

\section*{Introduction}

% Présentation du problème de vertex cover/du projet

\section{Structure de graphes}

\subsection{Structure de donnés}

% On détaille l'implémentation des graphes et on explique nos choix. Surtout pour éviter de répéter la même chose au début de la description de chaque algorithme.

\subsection{Génération aléatoire}

% On explique rapidement comment fonctionne nos algorithmes de génération et quel est leur intérêt. Le plus important va être d'expliquer nos choix et ce qui aurait changé sur la fonction de répartition si on fait autrement.

\section{Algorithmes réalisés}

\subsection{Algorithme glouton}

 
\subsubsection{Description}

% On explique le principe de l'algorithme, nos choix pour l'implémentation (ici, le tableau de degré par exemple), et on donne le pseudo-code.

\subsubsection{Complexité}

% Étude théorique de la complexité de l'algorithme. Éventuellement, quelle aurait été la complexité si on avait fait d'autres choix d'implémentation.

\subsubsection{Résultats}

% On chronomètre sur une batteries de graphes générés aléatoirement, on commente et on interprète. Je pense que cette partie est importante, c'est ce qui justifie l'implémentation de générateurs.

\subsection{Pour les arbres}

\subsubsection{Description}

\subsubsection{Complexité}

\subsubsection{Résultats}


\subsection{Généralisation aux graphes quelconques}

% On s'écarte un peu de l'ordre des questions, mais je pense que c'est mieux de parler de l'approximation basée sur le parcours en profondeur directement après l'algorithme sur les arbres.

\subsubsection{Description}

\subsubsection{Complexité}

\subsubsection{Résultats}


\subsection{Pour les graphes bipartis}

\subsubsection{Description}

\subsubsection{Complexité}

\subsubsection{Résultats}


\subsection{Un algorithme 2-approché}
\ 
\subsubsection{Description}

\subsubsection{Complexité}

\subsubsection{Résultats}


\subsection{Algorithme paramétrique pour faible couverture}

\subsubsection{Description}

\subsubsection{Complexité}

\subsubsection{Résultats}


\section{Résolution avec un sat-solver}

% Proposition de plan qui va sûrement évoluer quand on se sera penchés sur cette partie.

\subsection{Réduction}

\subsection{Algorithme et complexité}

% Je sais pas si on aura grand chose à dire sur notre implémentations des algorithmes de réduction, mais leur complexité et surtout la complexité du problème sat induit sont intéressantes.

\subsection{Résultats}

\end{document}
